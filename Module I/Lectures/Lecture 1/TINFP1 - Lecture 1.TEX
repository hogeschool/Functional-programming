\documentclass{beamer}
\usetheme{Warsaw}
\usepackage{hrslides}
\usepackage{graphicx}
\usepackage[space]{grffile}
\usepackage{listings}
\lstset{language=C,
basicstyle=\ttfamily\footnotesize,
frame=shadowbox,
mathescape=true,
breaklines=true}
\usepackage[utf8]{inputenc}

\title{Functional programming introduction}

\author{Dr. Giuseppe Maggiore}

\institute{Hogeschool Rotterdam \\ 
Rotterdam, Netherlands}

\date{}

\begin{document}
\maketitle

\SlideSection{Introduction}
\SlideSubSection{Course introduction}
\begin{slide}{
\item This course is about \textit{functional programming} (FP)
\item FP is a programming paradigm that focuses on data transformation instead of memory manipulation
}\end{slide}

\SlideSubSection{Other paradigms}
\begin{slide}{
\item Imperative programming (JavaScript, Java)
\item OO programming (Java)
\item Declarative programming (SQL)
}\end{slide}

\SlideSubSection{Imperative programming}
\begin{slide}{
\item Most used, most important programming tool nowadays
\item Strongly connected to hardware
\item CPU and memory are inherently imperative
}\end{slide}

\begin{slide}{
\item Mostly focused on the notion of \textit{destructively updating \textbf{state}}
\item \texttt{x = y;} loses (destroys) the previous value of \texttt{x}
}\end{slide}

\SlideSubSection{The risks of state}
\begin{slide}{
\item State and references cause undesired behaviours
\item Each thread/function/class has pointers to shared state
\item Each thread/function/class makes locally innocent modifications that as a whole break everything
}\end{slide}

\begin{frame}[fragile]
\begin{lstlisting}
money = 10000eur

putMoney m =
  newMoney := money + m
  money    := newMoney
  
takeMoney m =
  if money > m then
    newMoney := money - m
    money    := newMoney
    return OK
  else
    return NOT_ENOUGH_BALANCE
\end{lstlisting}
\end{frame}

\begin{slide}{
\item \texttt{runParallel(takeMoney 10000, takeMoney 10000)} may return OK
}\end{slide}

\begin{frame}[fragile]
\begin{lstlisting}
for a in asteroids
  if collision(a, projectiles)
    remove(a, asteroids)
  
for p in projectiles
  if collision(p, asteroids)
    remove(p, projectiles)
\end{lstlisting}
\end{frame}

\begin{slide}{
\item Will never remove anything from \texttt{projectiles}!
}\end{slide}

\begin{frame}[fragile]
\begin{lstlisting}
static missileConnection = connect("192.168.1.1//nuke")

class EmployeePayroll =
  DoNotPayProgrammerBonuses : () -> () =
    missileConnection.Launch()
\end{lstlisting}
\end{frame}

\begin{slide}{
\item Will cause WW3 if programmer bonuses are not paid
}\end{slide}

\begin{slide}{
\item Concurrent updates are dangerous
\item Creation of ``half processed'' states of the program which make no sense
}\end{slide}

\begin{slide}{
\item Functions that return \texttt{void} may really do anything
\item A function signature is meaningless
\item We cannot prevent wrong composition of functions with libraries
}\end{slide}


\SlideSubSection{Referential transparency}
\begin{slide}{
\item One of the core tenets of FP
\item Same input always yields same output
\item Predictable code
}\end{slide}

\begin{slide}{
\item How do we get this?
\item We forbid...
\pause
\item MUTABLE VALUES!!!!!
}\end{slide}

\begin{slide}{
\item Easier testing (only input values, not function order)
\item Stronger encapsulation (less dependencies from order)
\item Less chances of access mistakes (no way to access unrelated stuff)
}\end{slide}

\SlideSubSection{Introduction to ML}
\begin{slide}{
\item FP is very old
\item First high-level programming language, in the 50's, was LISP (FP)
\item Never been particularly popular
\begin{itemize}
\item It makes programming more difficult
\item It \textit{seems} to yield better code on average
\end{itemize}
}\end{slide}

\begin{slide}{
\item We will use a dialect of ML (Meta-Language)
\item Early 70's
\item Hybrid programming language which discourages, but does not forbid, value mutation
}\end{slide}

\begin{slide}{
\item We will use F\#
\item Originally built by Microsoft
\item Now fully open sourced
\item Best debugger, tool, and library support of any other FP language
\item Most other FP languages have ``academic quality'' (that is not very high) of tooling
}\end{slide}

\SlideSubSection{\texttt{let}}
\begin{frame}[fragile]
\begin{lstlisting}
let (i:int) = 100

let (x:float) = 10.0

let (b:bool) = true
\end{lstlisting}
\end{frame}

\begin{frame}[fragile]
\begin{lstlisting}
let f (i:int) (j:int) : int = i + j
\end{lstlisting}
\end{frame}


\begin{frame}[fragile]
\begin{lstlisting}
let quadratic (a:float) (b:float) (c:float) (x:float) : float =
  let t0 = c
  let t1 = x * b
  let t2 = x * x * a
  t0 + t1 + t2
\end{lstlisting}
\end{frame}


\begin{frame}[fragile]
\begin{lstlisting}
let distance (x1:float,y1:float) (x2:float,y2:float) : float = 
  let dx = x1-x2
  let dy = y1-y2
  System.Math.Sqrt(dx * dx + dy * dy)
\end{lstlisting}
\end{frame}


\SlideSubSection{\texttt{if-then-else}}
\begin{frame}[fragile]
\begin{lstlisting}
let tooCloseToZero (p:float * float) : bool =
  if distance (0.0,0.0) p < 10.0 then
    true
  else
    false
\end{lstlisting}
\end{frame}

\SlideSubSection{\texttt{match-with}}
\begin{frame}[fragile]
\begin{lstlisting}
let someNumbers (i:int) : string =
  match i with
  | 0 -> "zero"
  | 1 -> "one"
  | 3 -> "three"
  | 5 -> "five"
  | 7 -> "seven"
  | _ -> "number I don't like"
\end{lstlisting}
\end{frame}

\SlideSubSection{\texttt{let rec}}
\begin{frame}[fragile]
\begin{lstlisting}
let rec fac (n:int) : int = 
  if n = 0 then 
    1
  else
    n * fac(n-1)
\end{lstlisting}
\end{frame}

\begin{frame}[fragile]
\begin{lstlisting}
let rec fac (n:int) : int = 
  match n with
  | 0 -> 1
  | _ -> n * fac(n-1)
\end{lstlisting}
\end{frame}

\begin{frame}[fragile]
\begin{lstlisting}
let rec fac : int -> int = 
  function
  | 0 -> 1
  | n -> n * fac(n-1)
\end{lstlisting}
\end{frame}


\SlideSubSection{Conclusions and assignment}
\begin{slide}{
\item FP is a powerful programming paradigm
\item Very different from imperative programming
\item Functions are just ``data pumps''
\begin{itemize}
\item Data goes in
\item No side-effects
\item Data goes out
\end{itemize}
}\end{slide}

\begin{slide}{
\item The assignments are on Natschool
\item \textbf{Restore} the games to a working state
\item Hand-in a \textbf{printed} report that only contains your \textbf{sources} and the associated \textbf{documentation}
}\end{slide}

\begin{slide}{
\item Any book on the topic will do
\item I did write my own (Friendly F\#) that I will be loosely following for the course, \textbf{but it is absolutely not mandatory or necessary to pass the course}
}\end{slide}


\begin{frame}{Dit is het}
\center
\fontsize{18pt}{7.2}\selectfont
The best of luck, and thanks for the attention!
\end{frame}

\end{document}


\begin{slide}{
\item Some data may be sent with less bits, as it changes less
\item Some data may be sent in keyframes and incremental frames, if it changes incrementally
\item If we know that datagrams have a certain statistical distribution, we can use Huffman encoding or other compression mechanisms
}\end{slide}

\begin{frame}{\CurrentSection}
\center
\includegraphics[height=5cm]{Pics/Huffman.png}
\end{frame}

\begin{frame}[fragile]
\begin{lstlisting}

\end{lstlisting}
\end{frame}
