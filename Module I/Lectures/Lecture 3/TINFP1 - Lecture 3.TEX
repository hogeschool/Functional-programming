\documentclass{beamer}
\usetheme{Warsaw}
\usepackage{hrslides}
\usepackage{graphicx}
\usepackage[space]{grffile}
\usepackage{listings}
\lstset{language=C,
basicstyle=\ttfamily\footnotesize,
frame=shadowbox,
mathescape=true,
showstringspaces=false,
showspaces=false,
breaklines=true}
\usepackage[utf8]{inputenc}

\title{Lists and higher-order functions}

\author{Dr. Giuseppe Maggiore}

\institute{Hogeschool Rotterdam \\ 
Rotterdam, Netherlands}

\date{}



\begin{document}
\maketitle

\SlideSection{Introduction}
\SlideSubSection{Lecture topics}
\begin{slide}{
\item Recursive definition of lists as union types
\item Recursive traversal of lists with pattern matching
\item Generalization of lists to multiple data types
\item Generalization of list traversals
}\end{slide}

\SlideSubSection{Definition and basic traversals}
\begin{frame}[fragile]
\begin{lstlisting}
type IntList =
  | Empty
  | Cons of int * IntList
\end{lstlisting}
\end{frame}

\begin{frame}[fragile]
\begin{lstlisting}
let l = Cons(0, Cons(1, Cons(2, Empty)))
\end{lstlisting}
\end{frame}

\begin{frame}[fragile]
\begin{lstlisting}
let rec listBetween l u =
  if l > u then Empty
  else
    Cons(l, listBetween (l+1) u)
\end{lstlisting}
\end{frame}

\begin{frame}[fragile]
\begin{lstlisting}
let (++) h t = Cons(h, t)
\end{lstlisting}
\end{frame}

\begin{frame}[fragile]
\begin{lstlisting}
let l' = 0 ++ (1 ++ (2 ++ Empty))
\end{lstlisting}
\end{frame}

\begin{frame}[fragile]
\begin{lstlisting}
let rec listBetween' l u =
  if l > u then Empty
  else
    l ++ listBetween' (l+1) u
\end{lstlisting}
\end{frame}

\begin{frame}[fragile]
\begin{lstlisting}
let rec length l =
  match l with
  | Empty -> 0
  | Cons(h,t) -> 1 + length t
\end{lstlisting}
\end{frame}

\begin{frame}[fragile]
\begin{lstlisting}
let rec sum l =
  match l with
  | Empty -> 0
  | Cons(h,t) -> h + sum t
\end{lstlisting}
\end{frame}

\begin{frame}[fragile]
\begin{lstlisting}
let rec product l =
  match l with
  | Empty -> 1
  | Cons(h,t) -> h * sum t
\end{lstlisting}
\end{frame}

\begin{frame}[fragile]
\begin{lstlisting}
let rec add k l =
  match l with
  | Empty -> Empty
  | Cons(h,t) -> (h+k) ++ add k t
\end{lstlisting}
\end{frame}

\begin{frame}[fragile]
\begin{lstlisting}
let rec mul k l =
  match l with
  | Empty -> Empty
  | Cons(h,t) -> (h*k) ++ add k t
\end{lstlisting}
\end{frame}

\begin{frame}[fragile]
\begin{lstlisting}
let rec negate l =
  match l with
  | Empty -> Empty
  | Cons(h,t) -> (-h) ++ negate t
\end{lstlisting}
\end{frame}

\begin{frame}[fragile]
\begin{lstlisting}
let rec removeEven l =
  match l with
  | Empty -> Empty
  | Cons(h,t) when h % 2 = 0 -> removeEven t
  | Cons(h,t) -> h ++ removeEven t
\end{lstlisting}
\end{frame}

\begin{frame}[fragile]
\begin{lstlisting}
let rec removeOdd l =
  match l with
  | Empty -> Empty
  | Cons(h,t) when h % 2 = 1 -> removeOdd t
  | Cons(h,t) -> h ++ removeOdd t
\end{lstlisting}
\end{frame}

\begin{frame}[fragile]
\begin{lstlisting}
let rec minElem l =
  match l with
  | Empty -> failwith "Empty list has no minimum element"
  | Cons(h, Empty) -> h
  | Cons(h,t) -> min h (minElem t)
\end{lstlisting}
\end{frame}

\SlideSubSection{Generic lists}
\begin{slide}{
\item We have only seen lists of int's
\item How about lists of strings, float's, tuples?
\item How about lists of lists of tuples of lists?
\item How about ...
}\end{slide}

\begin{frame}[fragile]
\begin{lstlisting}
type List<'T> =
  | Empty
  | Cons of 'T * List<'T>
\end{lstlisting}
\end{frame}

\begin{frame}[fragile]
\begin{lstlisting}
let (++) h t = Cons(h, t)
\end{lstlisting}
\end{frame}

\begin{frame}[fragile]
\begin{lstlisting}
let rec length l =
  match l with
  | Empty -> 0
  | Cons(h,t) -> 1 + length t
\end{lstlisting}
\end{frame}

\begin{frame}[fragile]
\begin{lstlisting}
let rec sum l =
  match l with
  | Empty -> 0
  | Cons(h,t) -> h + sum t
\end{lstlisting}
\end{frame}

\SlideSubSection{Generic list traversals}
\begin{slide}{
\item There is a lot of repetition in the code above
\item We wish to reduce this repetition
\item We need to factor out the common parts, leaving the changing bits parameterized
\item The parameterization is done with ``functions as parameters'' (called \textit{higher order functions})
}\end{slide}

\begin{frame}[fragile]
\begin{lstlisting}
let rec map f l =
  match l with
  | Empty -> Empty
  | Cons(h,t) -> (f h) ++ map f t
\end{lstlisting}
\end{frame}

\begin{frame}[fragile]
\begin{lstlisting}
let add k l = map (fun x -> x + k) l
\end{lstlisting}
\end{frame}

\begin{frame}[fragile]
\begin{lstlisting}
let mul k l = map (fun x -> x * k) l
\end{lstlisting}
\end{frame}

\begin{frame}[fragile]
\begin{lstlisting}
let negate l = map (fun x -> -x) l
\end{lstlisting}
\end{frame}

\begin{frame}[fragile]
\begin{lstlisting}
let rec filter p l =
  match l with
  | Empty -> Empty
  | Cons(h,t) when p h -> h ++ filter p t
  | Cons(h,t) -> filter p t
\end{lstlisting}
\end{frame}

\begin{frame}[fragile]
\begin{lstlisting}
let removeEven l = filter (fun x -> x % 2 = 0) l
\end{lstlisting}
\end{frame}

\begin{frame}[fragile]
\begin{lstlisting}
let removeOdd l = filter (fun x -> x % 2 = 1) l
\end{lstlisting}
\end{frame}

\begin{frame}[fragile]
\begin{lstlisting}
let rec fold z f l = 
  match l with
  | Empty -> z
  | Cons(h,t) -> f h (fold z f t)
\end{lstlisting}
\end{frame}

\begin{frame}[fragile]
\begin{lstlisting}
let sum l = fold 0 (fun x y -> x + y) l
\end{lstlisting}
\end{frame}

\begin{frame}[fragile]
\begin{lstlisting}
let product l = fold 1 (fun x y -> x * y) l
\end{lstlisting}
\end{frame}

\begin{frame}[fragile]
\begin{lstlisting}
let rec reduce f l = 
  match l with
  | Empty -> failwith "Cannot reduce empty list"
  | Cons(h, Empty) -> h
  | Cons(h,t) -> fold h f t
\end{lstlisting}
\end{frame}

\begin{frame}[fragile]
\begin{lstlisting}
let minElem l = reduce (fun x y -> min x y) l
\end{lstlisting}
\end{frame}

\begin{frame}[fragile]
\begin{lstlisting}
let maxElem l = reduce (fun x y -> max x y) l
\end{lstlisting}
\end{frame}

\SlideSubSection{Generic generic list traversals}
\begin{slide}{
\item \texttt{fold} is the most powerful of the list traversals we have seen so far
\item It is so powerful that \texttt{map} and \texttt{filter} can be expressed as \texttt{fold}s
\item (\texttt{reduce} is already expressed as a \texttt{fold})
}\end{slide}

\begin{frame}[fragile]
\begin{lstlisting}
let map f l = fold Empty (fun h t -> Cons(f h,t)) l
\end{lstlisting}
\end{frame}

\begin{frame}[fragile]
\begin{lstlisting}
let filter p l = fold Empty (fun h t -> if p h then Cons(h,t) else t) l
\end{lstlisting}
\end{frame}

\SlideSubSection{Built-in lists and sequences}
\begin{slide}{
\item F\# has built-in lists and lazy lists
\item They have type \texttt{List<'a>} and \texttt{Seq<'a>}
\item Plus a huge library of combinator functions
\item Plus shortcut syntax for some combinators
}\end{slide}

\begin{frame}[fragile]
\begin{lstlisting}
let l = [1..100]
\end{lstlisting}
\end{frame}

\begin{frame}[fragile]
\begin{lstlisting}
let l = List.map (fun i -> i * i) [1..100]
\end{lstlisting}
\end{frame}

\begin{frame}[fragile]
\begin{lstlisting}
let l =
  [
    for i = 1 to 100 do 
    yield i * i
  ]
\end{lstlisting}
\end{frame}

\begin{frame}[fragile]
\begin{lstlisting}
let l = List.filter (fun i -> i % 2 = 0) (List.map (fun i -> i * i) [1..100])
\end{lstlisting}
\end{frame}

\begin{frame}[fragile]
\begin{lstlisting}
let l = [1..100] |> List.map (fun i -> i) |> List.filter (fun i -> i % 2 = 0)
\end{lstlisting}
\end{frame}

\begin{frame}[fragile]
\begin{lstlisting}
let l =
  [
    for i = 1 to 100 do 
      if i % 2 = 0 then
        yield i * i
  ]
\end{lstlisting}
\end{frame}

\begin{frame}[fragile]
\begin{lstlisting}
let l =
  [
    for i = 1 to 100 do 
    yield i * i
  ]
\end{lstlisting}
\end{frame}

\begin{frame}[fragile]
\begin{lstlisting}
let l =
  [
    for i = 1 to 100 do 
    for j = 1 to 100 do 
    yield i,j
  ]
\end{lstlisting}
\end{frame}

\begin{frame}[fragile]
\begin{lstlisting}
let l =
  [
    for i = 1 to 100 do 
      if i % 2 = 0 then
        for j = 1 to 100 do 
          if j > i then
            yield i,j
  ]
\end{lstlisting}
\end{frame}

\begin{frame}[fragile]
\begin{lstlisting}
let l =
  seq{
    for i = 1 to 100 do 
      if i % 2 = 0 then
        for j = 1 to 100 do 
          if j > i then
            yield i,j
  }
\end{lstlisting}
\end{frame}



\SlideSubSection{Conclusions and assignment}
\begin{slide}{
\item The assignments are on Natschool
\item \textbf{Restore} the games to a working state
\item Hand-in a \textbf{printed} report that only contains your \textbf{sources} and the associated \textbf{documentation}
}\end{slide}

\begin{slide}{
\item Any book on the topic will do
\item I did write my own (Friendly F\#) that I will be loosely following for the course, \textbf{but it is absolutely not mandatory or necessary to pass the course}
}\end{slide}


\begin{frame}{Dit is het}
\center
\fontsize{18pt}{7.2}\selectfont
The best of luck, and thanks for the attention!
\end{frame}

\end{document}
